\documentclass[11pt, oneside]{article}   	% use "amsart" instead of "article" for AMSLaTeX format
%\usepackage{geometry}                		% See geometry.pdf to learn the layout options. There are lots.
%\geometry{letterpaper}                   		% ... or a4paper or a5paper or ... 
%\geometry{landscape}                		% Activate for rotated page geometry
%\usepackage[parfill]{parskip}    		% Activate to begin paragraphs with an empty line rather than an indent

\usepackage{geometry}
 \geometry{
 a4paper,
 total={170mm,257mm},
 left=20mm,
 top=25mm,
 bottom=25mm
 }

\usepackage{graphicx}				% Use pdf, png, jpg, or eps§ with pdflatex; use eps in DVI mode
								% TeX will automatically convert eps --> pdf in pdflatex		
\usepackage{amssymb}
\usepackage{amsmath}
\usepackage{fancyhdr}
\usepackage[utf8]{inputenc}
\usepackage[english]{babel}
\usepackage{enumerate}
\usepackage{arcs}
\usepackage{cancel}
\usepackage{xfrac}
\usepackage{amsthm}
\usepackage{gensymb}
\usepackage{xspace}

\usepackage{epigraph}
\usepackage{csquotes}
\usepackage{soul}
\usepackage{subcaption}
\usepackage{verbatim}
\usepackage{hyperref}
%\usepackage{ctex}

%SetFonts

%SetFonts

\usepackage[inline]{asymptote}


\pagestyle{fancy}
\fancyhf{}
\lhead{\leftmark}

\title{Underwater Robotics -- Smooth Navigating with Depth Control}
\date{May 21, 2023}							% Activate to display a given date or no date

\newcommand{\latex}{\LaTeX\xspace}


\begin{document}
\maketitle

\section{Objectives}
We would like to design a robot that navigates underwater and satisfies the following requirements:
\begin{enumerate}
\item It should maintain a constant distance (i.e., 1 meter) from the bottom of water at any time.
\item It should maintain a constant speed moving forward/backward;
\item It should maintain self-balance while moving.
\end{enumerate}


\subsection{Building Parts}
\begin{enumerate}
\item A transparent plastic tube.
\item A raspberry pi board.
\item more $\cdots$ 
\end{enumerate}

\section{General Design}
Distance sensors, PID control, air adjustment, propellers, etc. (We will complete the design after discussions).

\section{Implementation}
We will implement the software in ROS2~\cite{doi:10.1126/scirobotics.abm6074}.

\section{Reference Project}
We would like to refer to this RC Submarine 4.0~\cite{ref:bec, ref:becv} project for guidance.

\bibliographystyle{plain}
\bibliography{reference}
 

\end{document} 